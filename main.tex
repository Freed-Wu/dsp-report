%&"main"
% Fakesection 检错
% Fakesubsubsection 宏包
\RequirePackage[l2tabu, orthodox]{nag}
% XXX: use nag at first to make all packages effecitive <03-11-19> %
% Fakesubsubsection 编译器
\RequirePackage{ifxetex}
\RequireXeTeX

\documentclass[twoside, openright]{report}
% Fakesection 基础
% Fakesubsection 文字
% Fakesubsubsection 颜色
\usepackage[x11names]{xcolor}
% XXX: conflict with microtype <02-11-19> %
% Fakesubsubsection 长度
\usepackage{printlen}
\uselengthunit{mm}
% Fakesubsubsection 效果
\usepackage{ulem}
% Fakesubsubsection 字体
\usepackage[no-math]{fontspec}
% XXX: use no-math to avoid conflict with mtxlatex <02-11-19> %
\setmainfont{Times New Roman}
% XXX: load ctex before siunitx to avoid \ohm ineffective <04-10-19> %
\usepackage[
	UTF8,
	fontset = windows,
	heading = true,
	zihao = -4,
	sub4section,
]{ctex}
\setCJKfamilyfont{zhsong}[
	AutoFakeBold = 2.17,
	AutoFakeSlant = 0.5,
]{SimSun}
\renewcommand*{\songti}{\CJKfamily{zhsong}}
% XXX: load newtxtext before textcomp to avoid option clash <04-10-19> %
\usepackage{newtxtext}
% Fakesubsubsection 字符边框
\usepackage{varwidth}
% Fakesubsection 断行
\usepackage{fvextra}
% Fakesubsection 标点
% XXX: load csquotes after fvextra to avoid warning <04-10-19> %
\usepackage{csquotes}
% Fakesubsubsection 中文引号
\DeclareQuoteStyle{zh}
{\symbol{"201C}}%“
{\symbol{"201D}}%”
{\symbol{"2018}}%‘
{\symbol{"2019}}%’
\setquotestyle{zh}
% Fakesubsubsection 日文引号
\DeclareQuoteStyle{ja}
{\symbol{"300E}}%『
{\symbol{"300F}}%』
{\symbol{"300C}}%「
{\symbol{"300D}}%」
% Fakesubsubsection 书名号
\DeclareQuoteStyle{book}
{\symbol{"300A}}%《
{\symbol{"300B}}%》
{\symbol{"3008}}%〈
{\symbol{"3009}}%〉
% Fakesubsection 样式
% Fakesubsubsection 目录
\usepackage{titletoc}
\titlecontents{chapter}[0pt]{\filright}{\contentspush{\thecontentslabel}}{}
{\titlerule*{.}\contentspage}
% Fakesubsubsection 章节
\ctexset{
	chapter = {
		name = ,
		number = \arabic{chapter},
		aftername = \hspace{1\ccwd},
		format = \ifthenelse{\value{chapter}=0}{\centering}{}
		\zihao{3}\heiti\bfseries,
		beforeskip = 0.5\ccwd,
		afterskip = 0.5\ccwd,
	},
	section = {
		aftername = \hspace{1\ccwd},
		format = \ifthenelse{\value{chapter}=0}{\centering}{}
		\zihao{-3}\heiti\bfseries,
		beforeskip = 0.5\ccwd,
		afterskip = 0.5\ccwd,
	},
	subsection = {
		format = \zihao{4}\heiti\bfseries,
	}
}
% Fakesubsubsection 图表
\usepackage{caption}
\captionsetup[figure]{labelsep=space}
\captionsetup[table]{labelsep=space}
\captionsetup{font=small}
\DeclareCaptionFont{blue}{\color{LightSteelBlue3}}
\setlength{\abovecaptionskip}{0.5\ccwd}
\setlength{\belowcaptionskip}{0.5\ccwd}
% Fakesubsubsection 子图表
\usepackage{subcaption}
% Fakesubsubsection 公式
\setlength{\abovedisplayskip}{0.5em}
\setlength{\belowdisplayskip}{0.5em}
% Fakesubsubsection 列表
\usepackage{enumitem}
\setlist[enumerate, 1]{
	fullwidth,
	label = (\arabic*),
	font = \textup,
	itemindent=2em
}
\setlist[enumerate, 2]{
	fullwidth,
	label = (\alph*),
	font = \textup,
	itemindent=4em
}
% Fakesubsubsection 代码
% XXX: need -shell-escape & pygmentize <04-10-19> %
\usepackage{minted}
\usepackage{boxie}
% XXX: conflict with fancybox <02-11-19> %
% Fakesubsubsection 问答
\usepackage{exercise}
\usepackage{tasks}
\renewcommand{\ExerciseName}{问题}
\renewcommand{\AnswerName}{回答}
\renewcommand{\listexercisename}{问题}
% Fakesubsubsection 改动

% Fakesection 插入
% Fakesubsection 表格
% Fakesubsubsection 三线
\usepackage{booktabs}
% Fakesubsubsection 对角线
\usepackage{diagbox}
% Fakesubsubsection 合并列
\usepackage{multicol}
% Fakesubsubsection 合并行
\usepackage{multirow}
% Fakesubsubsection 分割单元格
\usepackage{makecell}
% Fakesubsubsection 短表
\usepackage{tabu}
% Fakesubsubsection 长表
\usepackage{longtable}
% Fakesubsubsection 彩色表
\usepackage{colortbl}
\usepackage{tcolorbox}
\tcbuselibrary{skins}
\tcbuselibrary{breakable}
\tcbuselibrary{theorems}
\tcbuselibrary{listings}
\tcbuselibrary{xparse}
% XXX: need -shell-escape & pygmentize <04-10-19> %
\tcbuselibrary{minted}
% Fakesubsubsection 导入数据
\usepackage{csvsimple}
% Fakesubsection 图形
% Fakesubsubsection 插图
\usepackage{graphicx}
\graphicspath{{fig/}{fig/\arabic{chapter}/}}
% XXX: use '../' to make subfiles can find path <03-11-19> %
% Fakesubsubsection 环绕
\usepackage{wrapfig}
% Fakesubsubsection 图片重叠
\usepackage{overpic}
% Fakesubsubsection 徽标
\usepackage{hologo}
% Fakesubsubsection 条形码
\usepackage{ean13isbn}
% Fakesubsubsection 二维码
\usepackage{qrcode}
% Fakesubsection 符号
% Fakesubsubsection 数学符号
\usepackage{newtxmath}
\usepackage{bm}
% Fakesubsubsection 幻灯片符号
\usepackage{pifont}
% Fakesubsubsection 数学符号放缩
\usepackage{relsize}
% Fakesubsubsection 公式
\usepackage{cases}
\usepackage{physics}
% Fakesubsubsection 单位
\usepackage{siunitx}
\sisetup{mode=text}
% Fakesubsubsection 计算机
% Fakesubsection 媒体
\usepackage{media9}
% Fakesubsection 链接
\usepackage{hyperref}
\hypersetup{
	colorlinks = true,
	linkcolor = gray!50!black,
	citecolor = gray!50!black,
	backref = page
}
% Fakesubsection 批注
\usepackage{todonotes}
\usepackage{cooltooltips}
\usepackage{pdfcomment}
% Fakesubsection 文本框
\usepackage{boxedminipage2e}
% Fakesubsection 页眉页脚
\usepackage{fancyhdr}
\fancypagestyle{plain}{
	\pagestyle{fancy}
}

% Fakesection 设计
% Fakesubsection 水印
\usepackage{wallpaper}
% Fakesubsection 音乐
\usepackage{mtxlatex}
\mtxlatex
% Fakesubsection 主题

% Fakesection 布局
% Fakesubsection 页面
\usepackage{geometry}
% Fakesubsection 缩进
\usepackage{indentfirst}
% Fakesubsection 间距
\usepackage{setspace}
\usepackage[
	restoremathleading=false,
	UseMSWordMultipleLineSpacing,
	MSWordLineSpacingMultiple=1.5
]{zhlineskip}
% Fakesection 引用
% Fakesubsection 脚注
\renewcommand{\thefootnote}{\fnsymbol{footnote}}
\renewcommand{\thempfootnote}{\fnsymbol{mpfootnote}}
% Fakesubsection 引文
\usepackage{morewrites}
\usepackage[
	square, comma, numbers, super, sort&compress, longnamesfirst, sectionbib,
	nonamebreak
]{natbib}
% Fakesubsection 题注
\usepackage{epigraph}
% Fakesubsection 索引
\usepackage{makeidx}
\makeindex
% Fakesubsection 关联
% XXX: need amsmath <04-10-19> %
\numberwithin{Exercise}{chapter}
\numberwithin{Answer}{chapter}

% Fakesection 特殊功能
% Fakesubsection 页数统计
\usepackage{lastpage}
% Fakesubsection 数学表达式
\usepackage{calc}
% Fakesubsection 条件编译
\usepackage{subfiles}
% XXX: use subfiles at last to make all packages effecitive <03-11-19> %



\begin{document}

% Fakesection 扉页

\newcommand{\Title}{数字信号处理实验}

\begin{titlepage}
	\centering
	\begin{spacing}{1}
		\zihao{4}
		\vspace{0.5\ccwd}

		\vspace{1\ccwd}

		\includegraphics[width=7.41cm]{NJUST.ai}

		\vspace{0.2\ccwd}

		\fontsize{45pt}{45pt}\selectfont\heiti
		\Title

		\zihao{-1}
		\vspace{2\ccwd}
	\end{spacing}

	\begin{spacing}{1.5}
		\zihao{3}
		\begin{tabu} to 12.59cm{@{}X[c, 3.2cm]@{}X[c, 4cm]@{}X[c, 2.22cm]@{}X[c, 3.89cm]@{}}
			\textbf{作  者:} & \underline{\makebox[4cm][c]{\kaishu 吴振宇}} & \textbf{学 号:} & \underline{\makebox[3.89cm][c]{\kaishu 916101630117}} \\
			\textbf{学  院:} & \multicolumn{3}{c}{\underline{\makebox[10.11cm][c]{\kaishu 电子工程与光电技术学院}}} \\
			\textbf{专业(方向):} & \multicolumn{3}{c}{\underline{\makebox[10.11cm][c]{\kaishu 电子信息工程}}} \\
			\textbf{班  级:} & \multicolumn{3}{c}{\underline{\makebox[10.11cm][c]{\kaishu 电信4班}}} \\
			\textbf{题  目:} & \multicolumn{3}{c}{\underline{\makebox[10.11cm][c]{\kaishu}}} \\
			\textbf{} & \multicolumn{3}{c}{\underline{\makebox[10.11cm][c]{\kaishu}}}
		\end{tabu}
		\vspace{0em}
	\end{spacing}

	\begin{spacing}{1}
		\zihao{3}
		\vspace{3\ccwd}

		\zihao{-3}
		\textbf{指导者:}\underline{\makebox[15.5\ccwd][c]{}}

		\zihao{5}
		\hspace{5em}(姓名)\hspace{11em}(专业技术职务)

		\zihao{-3}
		    \underline{\makebox[15.5\ccwd][c]{}}

		\zihao{5}
		\hspace{5em}(姓名)\hspace{11em}(专业技术职务)

		\zihao{-3}
		\textbf{评阅者:}\underline{\makebox[15.5\ccwd][c]{}}

		\zihao{5}
		\hspace{5em}(姓名)\hspace{11em}(专业技术职务)

		\zihao{3}
		\vspace{2\ccwd}

		\zihao{-2}
		\number\year 年\number\month 月
	\end{spacing}
	\vspace{0em}
\end{titlepage}

% Fakesection 声明

\renewcommand{\abstractname}{\zihao{3}\heiti 声\hspace{2\ccwd}明}
\begin{abstract}
	\zihao{4}

	我声明,本\Title 及其研究工作和所取得的成果是本人在导师的指导下独立完成
	的。研究过程中利用的所有资料均已在参考文献中列出,其他人员或机构对
	本\Title 工作做出的贡献也已在致谢部分说明。

	本\Title 不涉及任何秘密,南京理工大学有权保存其电子和纸质文档,可以借阅
	或网上公布其部分或全部内容,可以向有关部门或机构送交并授权保存、借阅或网
	上公布其部分或全部内容。

	\vspace{2\ccwd}

	\begin{flushright}
		学生签名:\hspace{8em}

		\vspace{1\ccwd}

		年\hspace{3em}月\hspace{3em}日

		\vspace{2\ccwd}

		指导教师签名:\hspace{8em}

		\vspace{1\ccwd}

		年\hspace{3em}月\hspace{3em}日
	\end{flushright}
\end{abstract}

% Fakesection 摘要页眉页脚

\pagestyle{fancy}
\renewcommand{\headrulewidth}{0pt}
\fancyhead[LC, RC]{}
\fancyhead[LE, RO]{}
\fancyhead[RE, LO]{}
\fancyfoot[LC, RC]{}
\fancyfoot[LE, RO]{}
\fancyfoot[RE, LO]{}

% Fakesection 摘要

\newpage

\begin{center}
	\zihao{3}\renewcommand{\CJKglue}{\hskip 2pt}\heiti \Title 中文摘要

	\vspace{0.3em}

	\begin{boxedminipage}[][18cm]{\linewidth}
		\begin{spacing}{1.5}
			\zihao{-4}

			\vspace{1\ccwd}

			本文测量了FIR和IIR低通、带通、高通滤波器的幅频特性曲线,
			数据,并通过描点法绘制了,幅频特性曲线,并计算了相对误差
			。
			\cite{许爽2010综合设计性数字信号处理实验的探讨,郝小江基于,赵知劲2007数字信号处理实验}

			\vspace{2\ccwd}

			\noindent\textbf{关键词}\hspace{1\ccwd}FIR\hspace{1\ccwd}IIR

		\end{spacing}
	\end{boxedminipage}
\end{center}

% Fakesection 英文摘要

\newpage

\begin{center}
	\zihao{3}\renewcommand{\CJKglue}{\hskip 2pt}\heiti \Title 英文摘要

	\vspace{0.3em}

	\begin{boxedminipage}[][18cm]{\linewidth}
		\begin{spacing}{1.5}
			\zihao{3}

			\vspace{1em}

			\begin{tabu} to \linewidth{@{}X[l]@{}X[l, 8]@{}}
				\textbf{Title} &
				\hspace{2em}\underline{\makebox[15em][c]{\zihao{4}\songti Digital Signal Process}} \\
				\phantom{Title} &
				\hspace{2em}\underline{\makebox[15em][c]{\zihao{4}\songti }} \\
			\end{tabu}

			\textbf{Abstract}

			\zihao{-4}

			\vspace{2em}

			In this paper, the amplitude frequency characteristic
			curves of fir and IIR low-pass, band-pass and high-pass
			filters are measured, and the amplitude frequency
			characteristic curves are drawn by the point tracing
			method, and the relative errors are calculated. The
			conclusion is credible.

			\noindent\textbf{Keywords}\hspace{1em}FIR\hspace{1em}IIR

		\end{spacing}
	\end{boxedminipage}
\end{center}

% Fakesection 目录页眉页脚

\newpage

\renewcommand{\headrulewidth}{0.4pt}
\fancyhead[LC, RC]{\zihao{-2}\Title}
\fancyhead[LE, RO]{\zihao{5}第\thepage 页}
\fancyhead[RE, LO]{}

% Fakesection 目录

\pagenumbering{roman}

\setcounter{tocdepth}{2}

\renewcommand{\contentsname}{\zihao{3}\heiti 目次}
\tableofcontents
\listoffigures
\listoftables

\newpage
\mbox{}
\newpage

\listofexercises

\newpage
\mbox{}
\newpage

\pagenumbering{arabic}

% Fakesection 正文

\subfile{subfile/IIR}
\subfile{subfile/FIR}

% Fakesection 参考文献

\setcounter{chapter}{0}

\addcontentsline{toc}{chapter}{参考文献}

\bibliographystyle{gbt7714-plain}
\bibliography{bib/main}

\end{document}

